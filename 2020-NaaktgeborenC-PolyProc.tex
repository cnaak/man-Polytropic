%-----------------------------------------------------------------------------------------------
% 2020-NaaktgeborenC-PolyProc.tex - by C. Naaktgeboren
% License: CC-BY-NC-ND 4.0 - https://creativecommons.org/licenses/by-nc-nd/4.0/
%-----------------------------------------------------------------------------------------------
\documentclass[10pt,a4paper]{article}
%-----------------------------------------------------------------------------------------------
\usepackage[top=25mm,bottom=25mm,left=25mm,right=25mm]{geometry}
\usepackage{authblk}
\usepackage{pslatex}
\usepackage{graphicx}
\usepackage[squaren,cdot]{SIunits}
%-----------------------------------------------------------------------------------------------
\makeatletter
\immediate\write18{datelog > \jobname.info} % site script: $(date -u '+%Y-%m-%d %Hh%Mm%Ss UTC')
\makeatother
%-----------------------------------------------------------------------------------------------
\renewcommand\Affilfont{\itshape\small}
%-----------------------------------------------------------------------------------------------
\title{On Polytropic Processes}
\author[1]{C.~Naaktgeboren}
\affil[1]{%
    Universidade Tecnológica Federal do Paraná -- UTFPR, Câmpus Guarapuava.\par
    Grupo de Pesquisa em Ciências Térmicas.
}
\date{{\scriptsize\tt%
    \includegraphics[height=6.0mm]{cc/by-nc-nd.pdf}\\
    Compiled on \input{\jobname.info}
}}
%-----------------------------------------------------------------------------------------------
\begin{document}
%-----------------------------------------------------------------------------------------------

\maketitle

\begin{abstract}
    Work in progress.
    Here goes the abstract...
\end{abstract}

%-----------------------------------------------------------------------------------------------
\section{Introduction}

    Many engineering equilibrium thermodynamics processes  are  taken  to  follow  a  polytropic
    relationship of constant $Pv^n$, in which $P$ is the system pressure, in  \kilo\pascal,  $v$
    is the system specific volume, in $\meter\cubed\!\per\kilogram$, and $n$ is a  dimensionless
    polytropic  exponent~\cite{2013-CengelYA+BolesMA-AMGH};  which  means,  for  a  ``$1$--$2$''
    process, with end states labeled as ``1'' and ``2'',  the  relation  $P_1v_1^n  =  P_2v_2^n$
    holds true.

    Some maintream thermodynamics textbooks introduce polytropic processes  in  the  context  of
    closed system boundary work, as a  $P:P(v)$  relationship  to  plug  in  the  boundary  work
    integral,   which    contains    a    $P\,dv$    integrand~\cite{2013-CengelYA+BolesMA-AMGH,
    2002-MoranMJ+ShapiroHN-LTC, 1985-WylenG-Wiley}. In such texts, the  polytropic  relationship
    is frequently said to find support in measurents, while no specific  theoretical  derivation
    is presented at the point of introduction.

%-----------------------------------------------------------------------------------------------
\section*{Acknowledgments}

    This research received no specific grant from any funding agency in the public, private,  or
    not-for-profit sectors.

%-----------------------------------------------------------------------------------------------

\bibliographystyle{plain}
\bibliography{bibfile}

%-----------------------------------------------------------------------------------------------
\end{document}
%-----------------------------------------------------------------------------------------------
