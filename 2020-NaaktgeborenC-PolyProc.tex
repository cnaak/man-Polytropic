%-----------------------------------------------------------------------------------------------
% 2020-NaaktgeborenC-PolyProc.tex - by C. Naaktgeboren
% License: CC-BY-NC-ND 4.0 - https://creativecommons.org/licenses/by-nc-nd/4.0/
%-----------------------------------------------------------------------------------------------
\documentclass[10pt,a4paper]{article}
%-----------------------------------------------------------------------------------------------
\usepackage[top=25mm,bottom=25mm,left=25mm,right=25mm]{geometry}
\usepackage{authblk}
\usepackage{pslatex}
\usepackage{graphicx}
\usepackage[squaren,cdot]{SIunits}
\usepackage{multicol}
%-----------------------------------------------------------------------------------------------
\setlength{\columnsep}{8mm}
\setlength{\columnseprule}{0.4pt}
%-----------------------------------------------------------------------------------------------
\newtheorem{theorem}{Theorem}
\newtheorem{definition}{Definition}
\newtheorem{example}{Example}
%-----------------------------------------------------------------------------------------------
\makeatletter
\immediate\write18{datelog > \jobname.info} % site script: $(date -u '+%Y-%m-%d %Hh%Mm%Ss UTC')
\makeatother
%-----------------------------------------------------------------------------------------------
\renewcommand\Affilfont{\itshape\small}
%-----------------------------------------------------------------------------------------------
\title{On Exact and Local Polytropic Processes -- Requisites, Etymology, and Modeling}
\author[1]{C.~Naaktgeboren}
\affil[1]{%
    Universidade Tecnológica Federal do Paraná -- UTFPR, Câmpus Guarapuava.\par
    Grupo de Pesquisa em Ciências Térmicas.
}
\date{{\scriptsize\tt%
    \includegraphics[height=6.0mm]{cc/by-nc-nd.pdf}\\
    Compiled on \input{\jobname.info}
}}
%-----------------------------------------------------------------------------------------------
\begin{document}
%-----------------------------------------------------------------------------------------------

\maketitle

\begin{abstract}
    Work in progress.
    Here goes the abstract...
\end{abstract}

%-----------------------------------------------------------------------------------------------
\begin{multicols}{2}

%-----------------------------------------------------------------------------------------------
\section{Introduction}

    Many equilibrium engineering thermodynamics processes  are  taken  to  follow  a  polytropic
    relationship of constant $Pv^n$, in which $P$ is the system pressure, in  \kilo\pascal,  $v$
    is the system specific volume, in $\meter\cubed\!\per\kilogram$, and $n$ is a  dimensionless
    polytropic exponent~\cite{2013-CengelYA+BolesMA-AMGH}; which means, for a ``1--2''  process,
    with end states labeled as ``1'' and ``2'', the relation $P_1v_1^n = P_2v_2^n$ holds true.

    Some maintream thermodynamics textbooks introduce polytropic processes  in  the  context  of
    closed system boundary work, as a  $P:P(v)$  relationship  to  plug  in  the  boundary  work
    integral,   which    contains    a    $P\,dv$    integrand~\cite{2013-CengelYA+BolesMA-AMGH,
    2002-MoranMJ+ShapiroHN-LTC, 1985-WylenG-Wiley}. In such texts, the  polytropic  relationship
    is frequently said to find support in measurents, while no specific  theoretical  derivation
    is presented at the point of introduction.

    On  the  other  hand,  other  texts~\cite{1986-JonesJB+HawkinsGA-Wiley,   2006-BejanA-Wiley,
    2015-KroosKA+PotterMC-Cengage} include derivations that lead to a polytropic process, or  at
    least to an isentropic version of it, in which the exponent $n$ has a fixed value.

    Moreover, Bejan~\cite[p.~175]{2006-BejanA-Wiley} indicates that a constant  $Pv^n$  relation
    only holds \emph{locally} if the process is such that $n$ is a function of either $P$,  $v$,
    or both.

    A paper due to Christians~\cite{2012-ChristiansJ-IntJMechEngEduc} discusses the topic from a
    perspective  of  teaching  polytropic  processes  themselves,  placing   emphasis   on   the
    \emph{heat-to-work transfer ratio}---named by that author as ``energy transfer ratio''---and
    how its constancy not only yields, but constitutes a  pre-requisite  for  a  process  to  be
    polytropic, besides, naturally, the constancy of the caloric properties of the working  pure
    substance.

    This work develops the concepts of \emph{exact} and \emph{local} polytropic  processes,  and
    presents theory-derived \emph{requisites} for a process to be exactly polytropic.  Moreover,
    an  \emph{etymological}  discussion  is  presented  in  connection  to  the  usefulness   of
    \emph{local} polytropic processes for  generalized  equilibrium  engineering  thermodynamics
    process \emph{modeling}.

%-----------------------------------------------------------------------------------------------
\section{Exact and Local Polytropic Processes}

    In equilibrium engineering thermodynamics, a \emph{process}---more properly  a  quasi-static
    or quasi-equilibrium process---is defined in terms of changes  from  a  certain  equilibrium
    state of a system to  another~\cite{2013-CengelYA+BolesMA-AMGH},  with  process  \emph{path}
    being the (infinite) sequence of (quasi-)equilibrium states visited by the system during the
    process. A process can be referred to by its path, with implicit or explicit end states.

    It is worth noting that no constraints are stated for the end states of a ``process''.  This
    allows for the needed flexibility in describing the variety of  transformations  on  systems
    and control volumes possible in engineering thermodynamics.

    This lack of end state constraints in the definition of a process  allows  processes  to  be
    splitted into multiple sub-processes that still fit the definition of a process, as well as
    merged together into super-processes that also fit the definition of a process.

    In order  to  make  the  intended  distinction  between  proposed  ``exact''  and  ``local''
    polytropic processes, additional constraints need to be made to the process end states.  The
    following defines a \emph{logical process}, which is a process with constrained end states:

    \begin{definition}[logical process]\label{def:logical.proc}
        Define logical process as one in which its stated defining conditions, that  define  all
        of the allowed interactions for the underlying system or control volume, apply uniformly
        from, but not earlier than, its initial state until, but not later than, its end state.
    \end{definition}

    Therefore,   for   a   simple   compressible   system---admitting   only   work   and   heat
    interactions---either stated heat and work interactions, or system property  specifications,
    or combinations of the two, define possible logical processes.

    \begin{example}\label{ex:ideal.Otto}
        The well-known air-standard ideal Diesel power cycle with ``intake''  state  (of  lowest
        temperature and pressure) labeled as ``1'' can be divided in different ways  using  only
        logical processes. One such division is: (i)~``isentropic compression'', (ii)~``isobaric
        heating'',  (iii)~``isentropic  expansion'',  and  (iv)~``isochoric   cooling'',   which
        correspond to the ``1--2'', ``2--3'', ``3--4'', and  ``4--1''  canonical  processes  for
        this cycle,  respectively.  Other  possibilities  include:  (a)~``any  work  temperature
        increase'', and (b)~``non-compression pressure decrease'',  for  the  ``1--3''  (through
        ``2'') and ``3--1''  (through  ``4'')  processes,  respectively,  since  these  are  the
        farthermost end states that uniformly embrace the stated defining conditions.
    \end{example}

%-----------------------------------------------------------------------------------------------
\section*{Acknowledgments}

    This research received no specific grant from any funding agency in the public, private,  or
    not-for-profit sectors.

%-----------------------------------------------------------------------------------------------

\bibliographystyle{plain}
\bibliography{bibfile}

%-----------------------------------------------------------------------------------------------
\end{multicols}

%-----------------------------------------------------------------------------------------------
\end{document}
%-----------------------------------------------------------------------------------------------
