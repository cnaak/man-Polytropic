%-----------------------------------------------------------------------------------------------
% 2020-NaaktgeborenC-PolyProc.tex - by C. Naaktgeboren
% License: CC-BY-NC-ND 4.0 - https://creativecommons.org/licenses/by-nc-nd/4.0/
%-----------------------------------------------------------------------------------------------
\documentclass[10pt,a4paper]{article}
%-----------------------------------------------------------------------------------------------
\usepackage[top=25mm,bottom=25mm,left=25mm,right=25mm]{geometry}
\usepackage{authblk}
\usepackage{pslatex}
\usepackage{graphicx}
\usepackage[squaren,cdot]{SIunits}
%-----------------------------------------------------------------------------------------------
\makeatletter
\immediate\write18{datelog > \jobname.info} % site script: $(date -u '+%Y-%m-%d %Hh%Mm%Ss UTC')
\makeatother
%-----------------------------------------------------------------------------------------------
\renewcommand\Affilfont{\itshape\small}
%-----------------------------------------------------------------------------------------------
\title{On Exact and Local Polytropic Processes -- Requisites, Etymology, and Modeling}
\author[1]{C.~Naaktgeboren}
\affil[1]{%
    Universidade Tecnológica Federal do Paraná -- UTFPR, Câmpus Guarapuava.\par
    Grupo de Pesquisa em Ciências Térmicas.
}
\date{{\scriptsize\tt%
    \includegraphics[height=6.0mm]{cc/by-nc-nd.pdf}\\
    Compiled on \input{\jobname.info}
}}
%-----------------------------------------------------------------------------------------------
\begin{document}
%-----------------------------------------------------------------------------------------------

\maketitle

\begin{abstract}
    Work in progress.
    Here goes the abstract...
\end{abstract}

%-----------------------------------------------------------------------------------------------
\section{Introduction}

    Many equilibrium engineering thermodynamics processes  are  taken  to  follow  a  polytropic
    relationship of constant $Pv^n$, in which $P$ is the system pressure, in  \kilo\pascal,  $v$
    is the system specific volume, in $\meter\cubed\!\per\kilogram$, and $n$ is a  dimensionless
    polytropic  exponent~\cite{2013-CengelYA+BolesMA-AMGH};  which  means,  for  a  ``$1$--$2$''
    process, with end states labeled as ``1'' and ``2'',  the  relation  $P_1v_1^n  =  P_2v_2^n$
    holds true.

    Some maintream thermodynamics textbooks introduce polytropic processes  in  the  context  of
    closed system boundary work, as a  $P:P(v)$  relationship  to  plug  in  the  boundary  work
    integral,   which    contains    a    $P\,dv$    integrand~\cite{2013-CengelYA+BolesMA-AMGH,
    2002-MoranMJ+ShapiroHN-LTC, 1985-WylenG-Wiley}. In such texts, the  polytropic  relationship
    is frequently said to find support in measurents, while no specific  theoretical  derivation
    is presented at the point of introduction.

    On  the  other  hand,  other  texts~\cite{1986-JonesJB+HawkingsGA-Wiley,  2006-BejanA-Wiley,
    2015-KroosKA+PotterMC-Cengage} include derivations that lead to a polytropic process, or  at
    least to an isentropic version of it, in which the exponent $n$ has a fixed value.

    Moreover, Bejan~\cite[p.~175]{2006-BejanA-Wiley} indicates that a constant  $Pv^n$  relation
    only holds \emph{locally} if the process is such that $n$ is a function of either $P$,  $v$,
    or both.

    A paper due to Christians~\cite{2012-ChristiansJ-IntJMechEngEduc} discusses the topic from a
    perspective  of  teaching  polytropic  processes  themselves,  placing   emphasis   on   the
    \emph{heat-to-work transfer ratio}---named by that author as ``energy transfer ratio''---and
    how its constancy not only yields, but constitutes a  pre-requisite  for  a  process  to  be
    polytropic, besides, naturally, the constancy of the caloric properties of the working  pure
    substance.

    This work develops the concepts of \emph{exact} and \emph{local} polytropic  processes,  and
    presents theory-derived \emph{requisites} for a process to be exactly polytropic.  Moreover,
    an  \emph{etymological}  discussion  is  presented  in  connection  to  the  usefulness   of
    \emph{local} polytropic processes for  generalized  equilibrium  engineering  thermodynamics
    process \emph{modeling}.

%-----------------------------------------------------------------------------------------------
\section*{Acknowledgments}

    This research received no specific grant from any funding agency in the public, private,  or
    not-for-profit sectors.

%-----------------------------------------------------------------------------------------------

\bibliographystyle{plain}
\bibliography{bibfile}

%-----------------------------------------------------------------------------------------------
\end{document}
%-----------------------------------------------------------------------------------------------
